\documentclass[12pt]{article}
\usepackage[margin=1in]{geometry}
\usepackage{color}
\usepackage{tabto}
\usepackage{hyperref}
\hypersetup{colorlinks=true, urlcolor=blue}

\newcommand{\red}[1]{{\color{red}{#1}}}
\newcommand{\tabc}{4.3cm} %tab for course information
\newcommand{\tabg}{4.3cm} %tab for grade information

\title{Physics 121 - University Physics I: Mechanics \\ Fall 2017 \vspace{-2cm}}
\date{}

\begin{document}
\maketitle

\begin{center}
Cody Petrie \\ cody.petrie@mesacc.edu
\end{center}

\section*{Course Information}
Lecture: \tabto{\tabc} \red{Section 17904}, TTh 5:45-7:00 pm in \red{room} \\
Lab: \tabto{\tabc} \red{Section 17904}, TTh 7:15-8:30 pm in \red{room} \\
First Class: \tabto{\tabc} Tuesday 22 August \\
Last Class: \tabto{\tabc} Thursday 7 December \\
Mastering Physics ID: \tabto{\tabc} \red{ID here} \\
Prerequisites: \tabto{\tabc} Calculus I or department consent. One year of HS physics or PHY111 \\ and PHY112 is recommended.

\section*{Course Description}
Physics 121 is a combined lab-lecture class. This means that we will be discussing the theories, laws and math behind the physics and then reinforcing these ideas via experiments that we perform in the lab. Though the course is officially split up into two 75 minute parts, one for lecture and one for lab, we will be mixing the labs and lectures in the way that is best for learning the material.

\subsection*{MCCCD Official Course Competencies}
\begin{enumerate}
   \item Use fundamental physical laws and principles to solve problems encountered in academic and non-academic environments.
   \item Develop and use models that closely represent actual physical situations.
   \item Apply problem solving techniques in terms of logic, efficiency, and effectiveness.
   \item Work effectively in collaborative groups.
   \item Solve practical engineering and science problems.
\end{enumerate}

\subsection*{Textbook}
\textbf{Required:} Physics for Scientists and Engineers, a Strategic Approach second (or third) edition, Volume 1, by Randall D. Knight \\
\textbf{Optional:} Student Workbook: Physics for Scientists and Engineers second edition, Volume 1, by Randall D. Knight.

%\subsection*{Objectives}
%\begin{itemize}
%   \item Learn to use Newtonian kinematics to predict the motion of objects with translation and rotational motion.
%   \item Learn to use energy principles to predict properties of various physical systems.
%   \item Learn about Newton’s theory of gravitation.
%   \item Learn how to solve various type of physics problems individually and as a team.
%   \item If time permits we may learn the basics of oscillations and waves.
%\end{itemize}

\subsection*{Attendance Policy}
You will be expected to attend all sessions and actively participate. If you have three or more unofficial absences the instructor may withdraw you from the course (see MCC Student Handbook). Since you will learn most of the material by interacting with the instructor, your group members and classmates, missed material will be difficult to make up. You will have to take responsibility for your own learning. If you miss a classplease come and see me.

\section*{Class Format and Policies}
Course content and dates may vary from what is indicated here to meet the needs of this particular class. When changes occur an email will be sent to students. Make sure that you check often the email that is listed on Canvas as that is the email that will be used to disseminate information.

\subsection*{Labs}
Class will be held in a lab classroom so that we can take a break from lecture and practice what we are learning by performing a lab experiment. We will be completing a total of 13 labs by the end of the semester. Students will submit lab worksheets at the completion of each lab, though one lab assignment may be stretched between multiple class periods.

\subsection*{Homework}
There will be one homework assignment due each week on Thursday. The assignments will be based on material covered in class. The assignments will come from the Mastering Physics system. I will explain more about how to sign up for this system in class.

\subsection*{Quizzes}
There will be a quiz at the beginning of class every 1-3 classes. The quizzes will cover material from the most recent reading assignment and material that has already been covered in class.

\subsection*{Exams}
There will be two midterm exams and one final exam. The midterm exams will be mostly based on new material, however, due to the nature to physics the exams will be inherently cumulative. There will be one final exam during finals week that will be cumulative.

\subsection*{Grades}
\textbf{Grade Distribution:} \\
Labs \tabto{\tabg} 25\% \\
Homework \tabto{\tabg} 15\% \\
Quizzes \tabto{\tabg} 10\% \\
Midterm Exams \tabto{\tabg} 30\% (15\% each) \\
Final Exam \tabto{\tabg} 20\% \\

\textbf{Final Grade Scale:} \\
A \tabto{\tabg} $\ge$ 90\% \\
B \tabto{\tabg} 80\%-89.9\% \\
C \tabto{\tabg} 70\%-79.9\% \\
D \tabto{\tabg} 60\%-69.9\% \\
F \tabto{\tabg} $<$ 60\%

\subsection*{Late Work}
Generally late work will not be accepted. However I will make reasonable exceptions in extreme cases like family emergencies if I have been notified before the assignment is due. It is always better to ask me beforehand than to ask me afterwards.

\subsection*{Withdrawal}
See your student schedule in my.maricopa.edu for the Last Day to withdraw without an instructor signature for each class in which you are enrolled.

\subsection*{Disability Services}
Information for Students with Accommodation Needs:  If you have a documented disability (as protected by the Americans with Disability Act) or if you are pregnant or parenting (as protected under Title IX) and would like to discuss possible accommodations, please contact the MCC Disabilities Resources and Services Office at 480-461-7447 or email drsfrontdesk@mesacc.edu.

Access to Course Materials: If you are experiencing difficulty accessing course materials because of a disability please contact your instructor.  All students should have equal access to course materials and technology.

\subsection*{Academic Dishonesty}
Academic misconduct and dishonesty includes, but is not limited to, cheating, plagiarism, excessive absences, use of abusive or profane language, and disruptive and/or threatening behavior.  All instances of academic dishonesty will be reported to the Chair of the Physical Sciences Department and other appropriate authorities.  Students displaying acts of academic dishonesty are subject to grade adjustment, course failure, probation, suspension, or expulsion.  See the student handbook for more information regarding cases of academic misconduct.

\subsection*{Statement of Student Responsibilities}
It is your responsibility to understand the policies listed in this syllabus as these are the guidelines that your instructor will follow for grading, attendance, etc.  It is also your responsibility to read and understand the college policies included in the student handbook as they may apply to you in the case of an incomplete grade, withdraw for failure to attend, etc.
\\~\\
\href{https://www.mesacc.edu/sites/default/files/pages/section/students/student-life/MCC\%20Student\%20Handbook\%202014-2015-web.pdf}{MCC Student Handbook}

\subsection*{Tuition Charges and Refunds}
Students who officially withdraw from credit classes (in fall, spring, or summer) within the withdrawal deadlines listed below will receive a 100\% refund for tuition, class and registration processing fees. Deadlines that fall on a weekend or a college holiday will advance to the next college workday except for classes fewer than 10 calendar days in length or as specified by the college. Calendar days include weekdays and weekends. Refer to individual colleges for withdrawal and refund processes. Never attending is not an allowable refund exemption or an excuse of the debt incurred through registration.
\\~\\
{\it *Course fees and registration processing fees will be refunded only if the student qualifies for a 100\% refund. Debts owed to any MCCCD college must be satisfied before any refunds are paid to the student. Refunds for students receiving federal financial assistance are subject to federal guidelines. Requests for exceptions to the refund policy must be filed within one year from the semester in which the course was taken.}

\newpage
\section*{Fall 2017 Schedule}
\begin{tabular}{llll}
{\bf Date} & {\bf Reading} & {\bf Lab} & {\bf Topic} \\
22 Aug   & 1.1-3     & 1   & Intro; Motion Diagrams; x vs t \\
24 Aug   & 1.4-5     & 2   & Velocity; Acceleration \\
29 Aug   & 1.6-7,2.1 &     & 1D Motion; Problem Solving \\
31 Aug   & 2.2-4     & 3   & Free Fall   Inst. Vel.; Kinematic Eqs. \\
5 Sep    & 2.5-7     &     & Free Fall; Inclined Plane \\
7 Sep    & 3         & V   & Vectors \\
12 Sep   & 4.1-3     & 4   & 2D Acc; 2D Motion; Projectiles \\
         &           &     & Potential Exam time with 4.4 on 14th? \\
14 Sep   & 5.1-3     & 5   & Force \\
19 Sep   & 5.4-7     & 6?  & Newton's 1/2 Laws; Free-Body D. \\
21 Sep   & 6.1-3     &     & Equilibrium; N2L; Weight \\
26 Sep   & 6.4-5     & 7   & Contact Forces Friction; Drag \\
28 Sep   & 7.1-3     &     & Newton's 3rd Law \\
3 Oct    & 4.5-7     & 6   & Uniform Circular Motion; Nonuniform Circular Motion \\
5 Oct    & 8.1-3     &     & 2D Dynamics; Uniform Circular Motion; Circular Orbits \\
10 Oct   & 9.1-2     &     & Momentum and Impulse \\
12 Oct   & 9.3-5     & 9   & CoM; Inelastic Collisions; Explosions \\
17 Oct   & 10.1-3    & 11  & Conservation of Energy Energy; KE; (G)PE \\
19 Oct   & 10.4-6    &     & Hooke's Law; Elastic PE; Energy Diagrams \\
24 Oct   & 4.5-7     &     & Polar Coordinates; Nonuniform Circular Motion \\
26 Oct   & 11.1-3    &     & Work \\
31 Oct   & 11.4-6    &     & Work; PE \\
2 Nov    & 12.1-4    &     & Rotational Motion; Moment of Inertia \\
7 Nov    & 12.5-6    &     & Torque; Rotational Dynamics \\
9 Nov    & 12.7-8    &     & Static Equilibrium; Rolling \\
14 Nov   & 12.10-11  &     & Angular Momentum \\
16 Nov   & 13.1-3    &     & Isaac Newton; Law of Gravity \\
21 Nov   & 13.4-6    &     & g; G; GPE; Orbits \\
23 Nov   & N/A       &     & Thanksgiving \\
28 Nov   &           &     &  \\
30 Nov   &           &     &  \\
5 Dec    &           &     &  \\
7 Dec    &           &     &  \\
12 Dec   & N/A       &     & Final Exams \\
14 Dec   & N/A       &     & Final Exams
\end{tabular}

\end{document}
